% This is a basic Math Paper

\documentclass[11pt]{article}

% Preamble

\usepackage[margin=1in]{geometry}
\usepackage{amsfonts, amsmath, amssymb}
\usepackage{fancyhdr, float, graphicx}
\usepackage[utf8]{inputenc} % Required for inputting international characters
\usepackage[T1]{fontenc} % Output font encoding for international characters
\usepackage{fouriernc} % Use the New Century Schoolbook font
\usepackage[nottoc, notlot, notlof]{tocbibind}
\usepackage{url}

% Header and Footer
\pagestyle{fancy}
\fancyhead{}
\fancyfoot{}
\fancyhead[L]{\textit{\Large{HDPC Assignment 1}}}
%\fancyhead[R]{\textit{something}}
\fancyfoot[C]{\thepage}
\renewcommand{\footrulewidth}{1pt}



% Other Doc Editing
% \parindent 0ex
%\renewcommand{\baselinestretch}{1.5}

\begin{document}
	
	\begin{titlepage} 
		\centering 
		
		%---------------------------NAMES-------------------------------
		
		\huge\textsc{
			MIT World Peace University
		}\\
	
		\vspace{0.75\baselineskip} % space after Uni Name
		
		\LARGE{
			Human Dynamics and Peace in Communications\\
			First Year B. Tech, Trimester 3\\
			Academic Year 2021-22
		}
		
		\vfill % space after Sub Name
		
		%--------------------------TITLE-------------------------------
		
		\rule{\textwidth}{1.6pt}\vspace*{-\baselineskip}\vspace*{2pt}
		\rule{\textwidth}{0.6pt}
		\vspace{0.75\baselineskip} % Whitespace above the title
		
		
		
		\huge{\textsc{
				Assignment 1
			}} \\
		
		
		
		\vspace{0.5\baselineskip} % Whitespace below the title
		\rule{\textwidth}{0.6pt}\vspace*{-\baselineskip}\vspace*{2.8pt}
		\rule{\textwidth}{1.6pt}
		
		\vspace{1\baselineskip} % Whitespace after the title block

		%--------------------------SUBTITLE --------------------------	
			
		\LARGE\textsc{
			Experiment No. 1
		} % Subtitle or further description
		\vfill
		
		%--------------------------AUTHOR-------------------------------
		
		Prepared By
		\vspace{0.5\baselineskip} % Whitespace before the editors
		
		\Large{
			109054. Krishnaraj Thadesar\\
						
			Division 9 Batch I3
		}
		
		
		\vspace{0.5\baselineskip} % Whitespace below the editor list
		\today

	\end{titlepage}
	
\tableofcontents
\clearpage

\section{Explain the Different types of Communications. Give an Example of each.}	

\subsection{What are the types of Communication? }
There are four types of communication: verbal, nonverbal, written and visual. While many situations use one singular type of communication, you may find that some communications involve a blend of several different types at once. For example, sending an email involves only using written communication, but giving a presentation can involve all four types of communication.
\begin{enumerate}
	\item Verbal Communication:
		Verbal communication is the most common type of communication. It involves the use of spoken words or sign language to share information. Verbal communication can either happen face to face or through other channels, such as mobile phone, radio and video conferencing. Thus, if your job involves conducting business meetings, giving presentations and making phone calls, your employer would expect you to have good verbal communication skills.	
	\item Non Verbal Communication
		Nonverbal communication involves passive communication through the use of gestures, tone of voice, body language and facial expressions to share your thoughts and feelings. You can even communicate non-verbally by the way you dress. Nonverbal communication often supports or adds to verbal communication. For example, the tone of your voice and your posture can reveal your mood or emotions to those around you.
	\item Written Communication
		Written communication includes communicating through writing, typing or printing. It is done through channels such as letters, text messages, emails, social media and books. Businesses may prefer written communication because it has fewer chances of distortion. For example, communicating a business plan in writing ensures that everyone gets the same message and can refer to it any time in the future.
	\item Visual Communication
		Visual communication uses graphs, charts, photographs, maps and logos to share information. It is mostly used in combination with verbal or written communication in order to simplify the information. For example, using slides and flow charts during a presentation makes it easier for the audience to grasp complex data.
\end{enumerate}

\subsection{Examples}
	\begin{enumerate}
		\item Non Verbal Communication: \\
			Our body language, Sign language, facial expressions, etc. 
		\item Verbal Communication: \\
			Meetings, Phone Calls, Talking with Friends, Phones, radio, etc. 
		\item Written Communication\\
			Assignments, Memos, Letters, Newspapers etc. 
		\item Visual Communication\\
			Television, Videos, Logos, Posters, Art, etc. 	
\end{enumerate}

\clearpage

\section{What are the Barriers to Communication?}

Regardless of the type of communication: verbal, nonverbal, written, listening or visual, if we don't communicate effectively, we put ourselves and others at risk. Besides physical and technical barriers, there are six barriers to effective communication every employee and manager should strive to eradicate.\\

\subsection{Dissatisfaction or Disinterest with One's Job}

If you are unhappy or have lost interest in your job, you are far less likely to communicate effectively – both on the giving and receiving ends. In other words, your heart isn't in it.

\subsection{Inability to Listen to Others}
Active listening is an important aspect of effective communication. You cannot engage with someone if you are not listening to them because you will tend to make assumptions about their needs based on your perceptions versus reality.

\subsection{Lack of Transparency and Trust}
It is extremely difficult to communicate anything when there is a lack of transparency and trust. This is one of the worst communication barriers, as it is more of an abstract one, as opposed to the more practical ones on this list. It is often difficult to gain the trust of someone, and even more difficult to maintain that trust. Very few people can truly \textit{ever be transparent.} This leads to misunderstanding and mistrust, both of which are seldom easily resolved. 

\subsection{Communication Styles (when they differ)}
Everyone has their own communication style. Some people are very direct while others prefer a more indirect approach. Some use detailed data, while others rely on generalities, and so forth. Occasionally, one person is so entrenched in their way of communicating, they find it difficult to communicate with others who rely on a different style. You might hear comments such as, “Mary never explains what she wants me to do, she’s never specific” or “Bill gets so caught up in the weeds, that I lose focus on the bigger picture.”

\subsection{Conflicts in the Workplace}
Conflict can happen for a variety of reasons and when it does, it becomes a barrier to effective communication. The nature of the conflict is not necessarily important, what is important is working to resolve the conflict. When conflict is not eradicated, it grows and then people begin to take sides, which further impedes effective communication.

\subsection{Cultural Differences and Language}
As the world is getting more and more globalized, any large office may have people from several parts of the world. Different cultures have a different meaning for several basic values of society. Dressing, Religions or lack of them, food, drinks, pets, and the general behaviour will change drastically from one culture to another.

Hence it is a must that we must take these different cultures into account while communication. This is what we call being culturally appropriate. In many multinational companies, special courses are offered at the orientation stages that let people know about other cultures and how to be courteous and tolerant of others.

\subsection{How to overcome these barriers?}
\begin{enumerate}
	\item Be aware of language, message and tone
	\item Consult others before communication
	\item Communicate according to the need of receiver
	\item Consistency of Message
	\item Follow up Communication
\end{enumerate}


\end{document}