% This is a basic Math Paper

\documentclass[12pt]{article}

% Preamble

\usepackage[margin=1in]{geometry}
\usepackage{amsfonts, amsmath, amssymb}
\usepackage{fancyhdr, float, graphicx}
\usepackage[utf8]{inputenc} % Required for inputting international characters
\usepackage[T1]{fontenc} % Output font encoding for international characters
\usepackage{fouriernc} % Use the New Century Schoolbook font
\usepackage[nottoc, notlot, notlof]{tocbibind}
\usepackage{url}

% Header and Footer
\pagestyle{fancy}
\fancyhead{}
\fancyfoot{}
\fancyhead[L]{\textit{\Large{Physics Formulas}}}
%\fancyhead[R]{\textit{something}}
\fancyfoot[C]{\thepage}
\renewcommand{\footrulewidth}{1pt}



% Other Doc Editing
% \parindent 0ex
%\renewcommand{\baselinestretch}{1.5}

\begin{document}
	
	\begin{titlepage} 
		\centering 
		
		%---------------------------NAMES-------------------------------
		
		\huge\textsc{
			MIT World Peace University
		}\\
	
		\vspace{0.75\baselineskip} % space after Uni Name
		
		\LARGE{
			Maths\\
			First Year B. Tech, Trimester 3\\
			Academic Year 2021-22
		}
		
		\vfill % space after Sub Name
		
		%--------------------------TITLE-------------------------------
		
		\rule{\textwidth}{1.6pt}\vspace*{-\baselineskip}\vspace*{2pt}
		\rule{\textwidth}{0.6pt}
		\vspace{0.75\baselineskip} % Whitespace above the title
		
		
		
		\huge{\textsc{
        Polar Curve Tracing
        }} \\
		
		
		
		\vspace{0.5\baselineskip} % Whitespace below the title
		\rule{\textwidth}{0.6pt}\vspace*{-\baselineskip}\vspace*{2.8pt}
		\rule{\textwidth}{1.6pt}
		
		\vspace{1\baselineskip} % Whitespace after the title block

		%--------------------------SUBTITLE --------------------------	
			
		\LARGE\textsc{
			Notes
		} % Subtitle or further description
		\vfill
		
		%--------------------------AUTHOR-------------------------------
		
		Prepared By
		\vspace{0.5\baselineskip} % Whitespace before the editors
		
		\Large{
			109054. Krishnaraj Thadesar\\
			\vspace{1cm}
			Division 9 Batch I3
		}
		
		
		\vspace{0.5\baselineskip} % Whitespace below the editor list
		\today

	\end{titlepage}

\clearpage
\tableofcontents
\clearpage

\section{Tracing of Rose Curves}

\subsection{Rules}

\begin{enumerate}
    \item Symmetry\\
        same as polar curve tracing. 
    \item Pole
        Again same as polar. 
    \item Tangents at pole
        Again same as polar. 
    \item The curve \(r = a \sin n \theta\) or
    \(r = a \cos n \theta\)  consists: 
    \begin{enumerate}
        \item n equal loops if n is odd
        \item 2n equal loops if n is even
    \end{enumerate}
    \item For drawing the loops, divide each quadrant into n equal parts.\\
    \(r = a \sin n \theta\) or
    \(r = a \cos n \theta\) 
    
    \begin{enumerate}
        \item For sin first loop is drawn along $\theta = \frac{\pi/2}{n}$
         For cos first loop is drawn along $\theta = 0$
        \item If n is even draw loops in two sectors consecutively from $\theta = 0$ to $\theta = 2\pi$
        \item If n is odd, draw loops in the two sectors alternatively keepign two sectors between loops vacant.  
        \end{enumerate}

    \item Angle between radius vector and tangents.\\
    use the Formula $\tan\phi = \frac{r}{\frac{dr}{d\theta}}$ and find $\phi$\\
    Also find points where $phi = 0 \textit{ or } \infty $
    \item Prepare the table of values of r and $\theta$
    \begin{enumerate}
        \item $\sin n\theta,$ \\
        
        for $n\theta = 0, \pi, 2\pi, 4\pi \dots$\\
        $\implies \theta = 0, \frac{\pi}{n}, \frac{2\pi}{n}, \frac{3\pi}{n} $
        \item $\cos n \theta $\\ 
        
        for $n\theta = \frac{-\pi}{2}, \frac{\pi}{2}, \frac{3\pi}{2}, \frac{5\pi}{2}\dots $\\
        $ \theta = \frac{-\pi}{2n}, \frac{\pi}{2n}, \frac{3\pi}{2n}, \frac{5\pi}{2n}$
    \end{enumerate}
\end{enumerate}


\section{Numericals}

Q1. Trace the curve $r = a\sin 3\theta$
\begin{enumerate}
    \item If r is replaced by -r, and $\theta$ is replaced by $-\theta$
    $\implies$ the curve is symmetric about the perpendicular line passing through the pole that is 
    $\theta = \frac{\pi}{2} $
    \item For r = 0 and $\theta = 0$, the curve passes through the pole.  
\end{enumerate}

\section{Reduction Formula}

We willuse reduction formula to find the integration of examples like these. 

\[ \int_{}^{} \,dx \]


\end{document}